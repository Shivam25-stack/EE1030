\let\negmedspace\undefined
\let\negthickspace\undefined
\documentclass[journal]{IEEEtran}
\usepackage[a5paper, margin=10mm, onecolumn]{geometry}
%\usepackage{lmodern} % Ensure lmodern is loaded for pdflatex
\usepackage{tfrupee} % Include tfrupee package

\setlength{\headheight}{1cm} % Set the height of the header box
\setlength{\headsep}{0mm}     % Set the distance between the header box and the top of the text

\usepackage{gvv-book}
\usepackage{gvv}
\usepackage{cite}
\usepackage{amsmath,amssymb,amsfonts,amsthm}
\usepackage{algorithmic}
\usepackage{graphicx}
\usepackage{textcomp}
\usepackage{xcolor}
\usepackage{txfonts}
\usepackage{listings}
\usepackage{enumitem}
\usepackage{mathtools}
\usepackage{gensymb}
\usepackage{comment}
\usepackage[breaklinks=true]{hyperref}
\usepackage{tkz-euclide} 
\usepackage{listings}
% \usepackage{gvv}                                        
\def\inputGnumericTable{}                                 
\usepackage[latin1]{inputenc}                                
\usepackage{color}                                            
\usepackage{array}                                            
\usepackage{longtable}                                       
\usepackage{calc}                                             
\usepackage{multirow}                                         
\usepackage{hhline}                                           
\usepackage{ifthen}                                           
\usepackage{lscape}
\usepackage{amsmath}
\begin{document}

\bibliographystyle{IEEEtran}
\vspace{3cm}

\title{2011-XE-'53-65'}
\author{EE24BTECH11057 - SHIVAM SHILVANT*
}
% \maketitle
% \newpage
% \bigskip
{\let\newpage\relax\maketitle}

\renewcommand{\thefigure}{\theenumi}
\renewcommand{\thetable}{\theenumi}
\setlength{\intextsep}{10pt} % Space between text and floats


\numberwithin{equation}{enumi}
\numberwithin{figure}{enumi}
\renewcommand{\thetable}{\theenumi}

\begin{enumerate}
\setcounter{enumi}{9}

\item A plain carbon steel was annealed just above the eutectoid temperature. Microstructural analysis revealed that the proeutectoid ferrite content was 30 wt \% .The eutectoid reaction in the iron-iron carbide phase diagram is given below :

\begin{center}
    $\gamma\brak{0.76\ wt\% \ C}$ $\xrightleftharpoons[\text{heating}]{\text{cooling}}$ 
    $\alpha\brak{0.022\ \text{wt}\%\ \text{C}} + \text{Fe}  _3\text{C} \brak{6.7\ \text{wt}\% \ \text{C} }$
\end{center} 

The carbon content of the steel $\brak{\text{in wt}\%}$ is
\begin{multicols}{4}
\begin{enumerate}
    \item $0.24$
    \item $0.34$
    \item $0.44$
    \item $0.54$
\end{enumerate}
\end{multicols}

\item Match the materials in \textbf{Column-I}with the descriptions in \textbf{Column-II}.
\begin{table}[h]
    \centering
    \begin{tabular}{|c|c|} 
        \hline
        \textbf{Column-I} & \textbf{Column-II} \\
        \hline
        P. Zirconia & 1. Ultra-hard material \\
        \hline
        Q. Cubic boron nitride & 2. High temperature superconductor \\
        \hline
        R. Hafnium carbide & 3. Transformation toughening\\
        \hline
        S. Yttrium aluminium garnet & 4. Ultra-high temperature material \\
        \hline
         & 5. Host material for laser \\
        \hline
          & 6. Micro-crack toughening \\
        \hline
    \end{tabular}
    \label{tab:my_label}
\end{table}
\begin{multicols}{2}
\begin{enumerate}
    \item P-3, Q-4, R-1, S-2
    \item P-6, Q-1, R-4, S-2
    \item P-3, Q-1, R-4, S-5
    \item P-4, Q-6, R-1, S-5
\end{enumerate}
\end{multicols}

\item Match the materials in \textbf{Column-I}with the descriptions in \textbf{Column-II}.
\begin{table}[h]
    \centering
    \begin{tabular}{|c|c|} 
        \hline
        \textbf{Column-I} & \textbf{Column-II} \\
        \hline
        P. Polyacrylonitrile & 1. Hard and brittle material \\
        \hline
        Q. Nylon-6,6 & 2. Very high temperature resistant polymer \\
        \hline
        R. Polytetrafluoroethylene (PTFE) & 3. H-bonding\\
        \hline
        S. Ebonite & 4. Acrylic fibre \\
        \hline
         & 5. Rubber \\
        \hline
          &6. Polyester fibre \\
        \hline
    \end{tabular}
    \label{tab:my_label}
\end{table}
\begin{multicols}{2}
\begin{enumerate}
    \item P-6, Q-3, R-2, S-1
    \item P-2, Q-6, R-4, S-5
    \item P-4, Q-2, R-6, S-5
    \item P-4, Q-6, R-1, S-5
\end{enumerate}
\end{multicols}

\item Match the materials in \textbf{Column-I}with the descriptions in \textbf{Column-II}.
\begin{table}[h]
    \centering
    \begin{tabular}{|c|c|} 
        \hline
        \textbf{Column-I} & \textbf{Column-II} \\
        \hline
         P. Differential scanning calorimetry & 1. Residual stress measurement\\
        \hline
        Q. Atomic force microscopy & 2. Surface morphology of a material \\
        \hline
        R. Scanning electron microscopy & 3. Incident beam passes through a thin sample \\
        \hline
        S. X-ray diffraction & 4. Thermal expansion measurement \\
        \hline
         & 5. Resolution less than 1 nm is possible \\
        \hline
          & 6. Measurement of enthalpy change \\
        \hline
    \end{tabular}
    \label{tab:my_label}
\end{table}
\begin{multicols}{2}
\begin{enumerate}
    \item P-6, Q-5, R-2, S-1  
    \item P-4, Q-5, R-2, S-1
    \item P-4, Q-1, R-3, S-2 
    \item P-6, Q-1, R-5, S-3
\end{enumerate}
\end{multicols}

\item Match the materials in \textbf{Column-I}with the descriptions in \textbf{Column-II}.
\begin{table}[h]
    \centering
    \begin{tabular}{|c|c|} 
        \hline
        \textbf{Column-I} & \textbf{Column-II} \\
        \hline
         P. Thermal conductivity & 1. H m^{-1}\\
        \hline
        Q. Dielectric strength & 2. Wb m^{-2}\\
        \hline
        R. Magnetic permeability & 3. W m^{-1} K^{-1} \\
        \hline
        S. Capacitance & 4. V m^{-1} \\
        \hline
         & 5. C V^{-1} \\
        \hline
          & 6. J mol^{-1} K^{-1}\\
        \hline
    \end{tabular}
    \label{tab:my_label}
\end{table}
\begin{multicols}{2}
\begin{enumerate}
    \item P-6, Q-4, R-2, S-5
    \item P-3, Q-5, R-1, S-4
    \item P-3, Q-4, R-1, S-5
    \item P-6, Q-5, R-1, S-4
\end{enumerate}
\end{multicols}

\item It takes 4 h for carburising a steel at $900\degree\text{C} $. If the same carburising is to be accomplished in 2 h, what should be the temperature? The activation energy of diffusion of carbon in the steel is 151 kJ mol$^{-1}$.
\begin{multicols}{4}
\begin{enumerate}
    \item $850\degree\text{C}$
    \item $955\degree\text{C}$
    \item $1015\degree\text{C}$
    \item $1228\degree\text{C}$
\end{enumerate}
\end{multicols}

\item A steel specimen $\brak{12 \text{mm diameter and}\ 60 \ \text{mm length}}$ undergoes elastic deformation under tension. The deformed specimen experiences a longitudinal strain of 0.001. If the Poisson's ratio is 0.3, the diameter of the deformed specimen \brak{\text{in mm}} is 
\begin{multicols}{4}
\begin{enumerate}
    \item 12.0120
    \item 11.9964
    \item 11.9964
    \item 11.9880
\end{enumerate}
\end{multicols}

\textbf{Common Data Questions}\\

\textbf{Common Data for Questions 17 and 18:}\\
 The first peak in the powder X-ray diffraction pattern of an FCC metal appears at a Bragg angle of $19.2\degree$. The wavelength of Cu-K$_{\alpha}$ radiation used is 0.154 nm.\\
\item The lattice parameter of the metal \brak{\text{in nm}} is
\begin{enumerate}
\begin{multicols}{4}
    \item $ 0.4505$
    \item $ 0.4055$
    \item $ 0.3505$
    \item $ 0.3055$
    \end{multicols}
\end{enumerate}


\item The full width at half maximum \brak{\text{FWHM}} of the first peak is $0.35\degree$.
 Ignoring micro-strain and instrumental broadening, the crystallite size of the sample \brak{\text{in nm}} is 
\begin{enumerate}
\begin{multicols}{4}
    \item 20
    \item 24
    \item 200
    \item 240
\end{multicols}
\end{enumerate}

\textbf{Common Data for Questions 19 and 20:}\\
 For an intrinsic semiconductor, the mobilities of free electrons and holes are 0.14 m$^{2}$V$^{-1}$s$^{-1}$
 and 0.038 m$^{2}$V$^{-1}$s$^{-1}$, respectively. Its bandgap is 1.107 eV and electrical conductivity at 300 K is $3.99 \times 10^{-4} \Omega ^{-1}m^{-1}$.  \\
\item The free electron concentration \brak{\text{in m}^{-3}} at 300 K is
\begin{enumerate}
\begin{multicols}{4}
    \item $ 13.99 \times 10^{15}$
    \item $ 27.98 \times 10^{15}$
    \item $ 13.99 \times 10^{17}$
    \item $ 27.98 \times 10^{17}$
\end{multicols}
\end{enumerate}
\item What is the temperature at which the conductivity of the semiconductor is $0.399 \Omega^{-1}\text{m}^{-1}$?
\begin{multicols}{4}
\begin{enumerate}
    \item 343 K
    \item 443 K
    \item 493 K
    \item 543 K
\end{enumerate}
\end{multicols}

\textbf{Linked Answer Questions}\\
\textbf{Statement for Linked Answers Questions 21 and 22:}\\
A continuous and aligned glass fibre reinforced composite has a modulus of elasticity of 150 GPa in the longitudinal direction. The matrix is a polyester resin with a modulus of 4.5 GPa. The glass fibre has a modulus of 340 GPa.\\
\item The volume fraction of the glass fibres is
\begin{multicols}{4}
\begin{enumerate}
    \item $0.398$
    \item $0.434$
    \item $0.497$
    \item $0.566$
\end{enumerate}
\end{multicols}
\item If the cross-sectional area of the composite is 300
 mm$^{2}$, and a stress of 100 MPa is applied in the longitudinal direction, what will be the total load \brak{\text{in kN}} carried by the glass fibres?  
\begin{multicols}{2}
\begin{enumerate}
    \item $0.5$
    \item $5$
    \item $20.5$
    \item $29.5$
\end{enumerate}
\end{multicols}
\end{enumerate}
\end{document}
