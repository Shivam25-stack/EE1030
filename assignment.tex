%iffalse
\let\negmedspace\undefined
\let\negthickspace\undefined
\documentclass[journal,12pt,twocolumn]{IEEEtran}
\usepackage{cite}
\usepackage{amsmath,amssymb,amsfonts,amsthm}
\usepackage{algorithmic}
\usepackage{graphicx}
\usepackage{textcomp}
\usepackage{xcolor}
\usepackage{txfonts}
\usepackage{listings}
\usepackage{enumitem}
\usepackage{mathtools}
\usepackage{gensymb}
\usepackage{comment}
\usepackage[breaklinks=true]{hyperref}
\usepackage{tkz-euclide} 
\usepackage{listings}
\usepackage{gvv}                                        
%\def\inputGnumericTable{}                                 
\usepackage[latin1]{inputenc}                                
\usepackage{color}                                            
\usepackage{array}                                            
\usepackage{longtable}                                       
\usepackage{calc}                                             
\usepackage{multirow}                                         
\usepackage{hhline}                                           
\usepackage{ifthen}                                           
\usepackage{lscape}
\usepackage{tabularx}
\usepackage{array}
\usepackage{float}


\newtheorem{theorem}{Theorem}[section]
\newtheorem{problem}{Problem}
\newtheorem{proposition}{Proposition}[section]
\newtheorem{lemma}{Lemma}[section]
\newtheorem{corollary}[theorem]{Corollary}
\newtheorem{example}{Example}[section]
\newtheorem{definition}[problem]{Definition}
\newcommand{\BEQA}{\begin{eqnarray}}
\newcommand{\EEQA}{\end{eqnarray}}
\newcommand{\define}{\stackrel{\triangle}{=}}
\theoremstyle{remark}
\newtheorem{rem}{Remark}

% Marks the beginning of the document
\begin{document}
\bibliographystyle{IEEEtran}
\vspace{3cm}

\title{ASSIGNMENT-1}
\author{EE24BTECH11057-SHIVAM SHILVANT*}
\maketitle
\newpage
\bigskip

\renewcommand{\thefigure}{\theenumi}
\renewcommand{\thetable}{\theenumi}
\begin{enumerate}[start=9]
	\item The slope of tangent to curve $y=f\brak{x}$ at $\brak{x,f\brak{x}}$ is $2x + 1$.If the curve passes through the point $\brak{1,2}$, then the area bounded by curve, the $x$ axis and the line $x=1$ is
\hfill \brak{1995S}
\begin{enumerate}
    \item $\displaystyle\frac{5}{6}$\\ 
    \item $\displaystyle\frac{6}{5}$\\
    \item $\displaystyle\frac{1}{6}$\\ 
    \item $6$\\
\end{enumerate}
\item If $f\brak{x}=\displaystyle\frac{x}{\sin{x}}$ and $g\brak{x}=\displaystyle\frac{x}{\tan{x}}$, where$0<x\leq1$, then in this interval
	\hfill \brak{1997-2 Marks}
\begin{enumerate}
	\item both $f\brak{x}$ and $g\brak{x}$ are increasing functions.
	\item both $f\brak{x}$ and $g\brak{x}$ are decreasing functions.
	\item $f\brak{x}$ is an increasing function.
        \item $g\brak{x}$ is an increasing function.\\
\end{enumerate}

\item The function $f\brak{x}= \sin^4{x}+\cos^4{x}$ increases \\
	if \hfill \brak{1999-2 Marks}
\begin{enumerate}
    \item $0<x<\displaystyle\frac{\pi}{8}$\\
    \item $\displaystyle\frac{\pi}{4}<x<\displaystyle\frac{3\pi}{8}$\\
    \item $\displaystyle\frac{3\pi}{8} <x<\displaystyle\frac{5\pi}{8}$\\
    \item $\displaystyle\frac{5\pi}{8}<x<\displaystyle\frac{3\pi}{4}$\\
\end{enumerate}
\item Consider the following statements in S and \\
R \hfill \brak{2000S}\\
S: Both $\sin{x}$ and $\cos{x}$ are decreasing functions in the interval $\left(\displaystyle\frac{\pi}{2},\pi\right)$\\
R: If a differentiable function decreases in a interval $\brak{a,b}$,then its derivative also decreases in $\brak{a,b}$.\\
Which of the following is true ?
\begin{enumerate}
    \item Both S and R are wrong.
    \item Both S and R are correct, but R is not the correct explanation of S.
    \item S is correct and R is correct explanation for S.
    \item S is correct and R is wrong.\\
\end{enumerate}

\item Let $f\brak{x}=\int e^x\brak{x-1}\brak{x-2}dx$. Then $f$ decreases in the interval 
\hfill \brak{(2000S)}
\begin{enumerate}
	\item $\brak{-\infty,-2}$
	\item $\brak{-2,-1}$
	\item $\brak{1,2}$
	\item $\brak{2,+\infty}$\\
\end{enumerate}
\item If normal to curve $y=f\brak{x}$ at the point $\brak{3,4}$ makes an angle $\displaystyle\frac{3\pi}{4}$ with the positive $x$-axis, then $f$'\brak{3} = \hfill\brak{2000S}
\begin{enumerate}
    \item $-1$
    \item -$\displaystyle\frac{3}{4}$\\
    \item $\displaystyle\frac{4}{3}$
    \item $1$
\end{enumerate}
\item Let $f\brak{x}$=
$\begin{cases}
|x|, & \text{for}  0<|x| \leq 2\\ 
1, & \text{for}\  x=0
\end{cases}$then at $x=0$, \\$f$ has
\hfill \brak{2000S}
\begin{enumerate}
    \item a local maximum
    \item no local maximum
    \item a local minimum
    \item no extremum\\
\end{enumerate}
\item For all $x\in\brak{0,1}$
\hfill \brak{2000S}
\begin{enumerate}
    \item $e^x <1+x$
    \item $\log_e \brak{1+x} < x$
    \item$\sin{x} > x$
    \item$\log_e{x} > x $\\
\end{enumerate}
\item If $f\brak{x}=xe^{x\brak{1-x}}$,then $f\brak{x}$ is 
\hfill \brak{2001S}
\begin{enumerate}
    \item increasing on $\sbrak{-\displaystyle\frac{1}{2},1}$\\
    \item decreasing on R
    \item increasing on R
    \item decreasing on $\sbrak{-\displaystyle\frac{1}{2},1}$\\
\end{enumerate}

\item The triangle formed by tangent to curve $f\brak{x}=x^2+bx-b$ at the point $\brak{1,1}$ and the coordinate axes, lies in the first quadrant. If its area is $2$, then the value of $b$ is 
\hfill \brak{2001S}
\begin{enumerate}
    \item $-1$
    \item $3$ 
    \item $-3$
    \item $1$\\
\end{enumerate}
\item Let $f\brak{x}=\brak{1+b^2}x^2+2bx+1$ and let $m\brak{b}$ be the minimum value of $f\brak{x}$.As $b$ varies, the range of $m\brak{b}$ is 
\hfill \brak{2001S}
\begin{enumerate}
	\item $\sbrak{0,1}$\\
	\item $\lbrak{0},\rsbrak{\displaystyle\frac{1}{2}}$\\
	\item $\sbrak{\displaystyle\frac{1}{2},1}$\\
	\item $\lbrak{0},\rsbrak{1}$\\
\end{enumerate}
\item The length of a longest interval in which the function $3 \sin{x}-4\sin^3{x}$ is increasing,\\
is \hfill \brak{2002S}
\begin{enumerate}
    \item $\displaystyle\frac{\pi}{3}$\\
    \item $\displaystyle\frac{\pi}{2}$\\
    \item $\displaystyle\frac{3\pi}{2}$\\
    \item $\pi$
\end{enumerate}
\item The point\brak{s} on the curve $y^3+3x^2=12y$ where the tangent is vertical, is\brak{are}
\hfill \brak{2002S}\\
\begin{enumerate}
	\item $\brak{\pm \displaystyle\frac{4}{\sqrt{3}},-2}$\\
	\item $\brak{\pm\sqrt{\displaystyle\frac{11}{3}},1}$\\
	\item $\brak{0,0}$\\
	\item $\brak{\pm \displaystyle\frac{4}{\sqrt{3}},2}$\\\\
\end{enumerate}

\item In $\sbrak{0,1}$ Lagranges Mean Value theorem is NOT applicable to
\hfill \brak{2003S}

\begin{enumerate}
    \item $f\brak{x}$=
    $\begin{cases}
         \displaystyle\frac{1}{2}-x, x<\displaystyle\frac{1}{2}\\
	    \brak{\displaystyle\frac{1}{2}-x}^2, x\geq \displaystyle\frac{1}{2}
    \end{cases}$\\
    \item $f\brak{x}$=
    $\begin{cases}
	    \displaystyle\frac{\sin{x}}{x}, x\neq 0\\
        1, x=0
    \end{cases}$\\
    \item $f\brak{x}=x|x|$
    \item $f\brak{x}=|x|$\\
\end{enumerate}
\item Tangent is drawn to ellipse $\displaystyle\frac{x^2}{27}+y^2 = 1$ at $\brak{3\sqrt{3}\cos{\theta},\sin{\theta}}$ $\brak{\text{where}  \theta \in \brak{0,\displaystyle\frac{\pi}{2}}}$. Then the value of $\theta$ such that sum of intercepts on axes made by this tangent is minimum, is
\hfill \brak{2003S}
\begin{enumerate}
    \item $\displaystyle\frac{\pi}{3}$\\
    \item $\displaystyle\frac{\pi}{6}$\\
    \item $\displaystyle\frac{\pi}{8}$\\
    \item $\displaystyle\frac{\pi}{4}$\\
 \end{enumerate}
\end{enumerate}
\end{document}
