\let\negmedspace\undefined
\let\negthickspace\undefined
\documentclass[journal]{IEEEtran}
\usepackage[a5paper, margin=10mm, onecolumn]{geometry}
%\usepackage{lmodern} % Ensure lmodern is loaded for pdflatex
\usepackage{tfrupee} % Include tfrupee package

\setlength{\headheight}{1cm} % Set the height of the header box
\setlength{\headsep}{0mm}     % Set the distance between the header box and the top of the text

\usepackage{gvv-book}
\usepackage{gvv}
\usepackage{cite}
\usepackage{amsmath,amssymb,amsfonts,amsthm}
\usepackage{algorithmic}
\usepackage{graphicx}
\usepackage{textcomp}
\usepackage{xcolor}
\usepackage{txfonts}
\usepackage{listings}
\usepackage{enumitem}
\usepackage{mathtools}
\usepackage{gensymb}
\usepackage{comment}
\usepackage[breaklinks=true]{hyperref}
\usepackage{tkz-euclide} 
\usepackage{listings}
% \usepackage{gvv}                                        
\def\inputGnumericTable{}                                 
\usepackage[latin1]{inputenc}                                
\usepackage{color}                                            
\usepackage{array}                                            
\usepackage{longtable}                                       
\usepackage{calc}                                             
\usepackage{multirow}                                         
\usepackage{hhline}                                           
\usepackage{ifthen}                                           
\usepackage{lscape}
\usepackage{amsmath}
\begin{document}

\bibliographystyle{IEEEtran}
\vspace{3cm}

\title{2019-ST-'14-26'}
\author{EE24BTECH11057 - SHIVAM SHILVANT*
}
% \maketitle
% \newpage
% \bigskip
{\let\newpage\relax\maketitle}

\renewcommand{\thefigure}{\theenumi}
\renewcommand{\thetable}{\theenumi}
\setlength{\intextsep}{10pt} % Space between text and floats


\numberwithin{equation}{enumi}
\numberwithin{figure}{enumi}
\renewcommand{\thetable}{\theenumi}

\begin{enumerate}


\item A fair die is rolled two times 
independently. Given that the outcome on the first roll is 1, the expected value of the sum of the two outcomes is:  

\begin{multicols}{4}
\begin{enumerate}
    \item $4$
    \item $4.5$
    \item $3$
    \item $5.5$
\end{enumerate}
\end{multicols}

\item The dimension of the vector space of $7\times7$
 real symmetric matrices with trace zero and the sum of the off-diagonal elements zero is:   .
\begin{multicols}{4}
\begin{enumerate}
    \item 47
    \item 28
    \item 27
    \item 26
\end{enumerate}
\end{multicols}

\item Let A be a $6\times6$
 complex matrix with $\text{A}^{3}\neq0$ and $\text{A}^{4}=0$. Then the number of Jordan blocks of A is:  

\begin{multicols}{4}
\begin{enumerate}
    \item 1 or 6
    \item 2 or 3
    \item 4
    \item 5
\end{enumerate}
\end{multicols}

\item Let $X_{1}, ..., X_{n}$ be a random sample from a uniform distribution defined over $\brak{0,\theta}$, where $\theta>0$ and $n\geq2$. Let X\brak{1}=min$\cbrak{X_{1}, ..., X_{n}}$ and X\brak{n}=max$\cbrak{X_{1}, ..., X_{n}}$. Then the covariance between X\brak{n} and X\brak{1}/X\brak{n} is:
\begin{multicols}{4}
\begin{enumerate}
    \item 0
    \item $n\brak{n+1}\theta$
    \item $n\theta$
    \item $n^2\brak{n+1}\theta$
\end{enumerate}
\end{multicols}

\item Let $X_{1}, ..., X_{n}$be a random sample drawn from a population with probability density function f\brak{x;\theta}=$\theta x^{\theta-1}, 0\leq x\leq1,\theta>0$. Then the maximum likelihood estimator of $\theta$ is:
\begin{multicols}{4}
\begin{enumerate}
    \item $\frac{-n}{\sum_{i=1}^n \log_e {X_i}}$
    \item $\frac{-\sum_{i=1}^n \log_e {X_i}}{n}$
    \item $\brak{\Pi_{i=1}^n X_i}^{1/n}$
    \item $\frac{\brak{\Pi_{i=0}^n X_i}}{n}$
\end{enumerate}
\end{multicols}

\item Let $Y_i$ = $\beta_0+\beta_1x_{1i}+\beta_2x_{2i}+\epsilon_i$, for $i=1, ... ,10$, where $x_{1i}'s \ \text{and}\ x_{2i}'s$ are fixed covariates and $\epsilon_i$'s are uncorrelated random variables with mean 0 and unknown variance $\sigma^{2}$. Here $\beta_0,\ \beta_1,\ \text{and}\ \beta_2$ rare unknown parameteres. Further, define $\hat{Y_i} = \hat{\beta_0}  \ +\hat{\beta_1}x_{1i} \  +\hat{\beta_2}x_{2i}$, where $\brak{\hat{\beta_0},\hat{\beta_1},\hat{\beta_2}}$ is the unbiased least square estimator of $\brak{\beta_0,\beta_1,\beta_2}$. Then an unbiased estimator of $\sigma^2$ is
\begin{multicols}{4}
\begin{enumerate}
    \item $\frac{\sum_{i=1}^10 \brak{Y_i-\hat{Y_i}}^2}{10}$
    \item $\frac{\sum_{i=1}^10 \brak{Y_i-\hat{Y_i}}^2}{7}$
    \item $\frac{\sum_{i=1}^10 \brak{Y_i-\hat{Y_i}}^2}{8}$
    \item $\frac{\sum_{i=1}^10 \brak{Y_i-\hat{Y_i}}^2}{9}$
\end{enumerate}
\end{multicols}

\item For $i = 1, 2, 3$, let $Y_{i} = \alpha + \beta x_{i} + \epsilon_{i}$, where $x_{i}$'s are fixed covariates, and $\epsilon_{i}$'s are independent and identically distributed standard normal random variables. Here, $\alpha$ and $\beta$ are unknown parameters. Given the following observation,
\\
\begin{table}[h]
    \centering
    \begin{tabular}{|c|c|} 
        \hline
        \textbf{Column-I} & \textbf{Column-II} \\
        \hline
        P. Zirconia & 1. Ultra-hard material \\
        \hline
        Q. Cubic boron nitride & 2. High temperature superconductor \\
        \hline
        R. Hafnium carbide & 3. Transformation toughening\\
        \hline
        S. Yttrium aluminium garnet & 4. Ultra-high temperature material \\
        \hline
         & 5. Host material for laser \\
        \hline
          & 6. Micro-crack toughening \\
        \hline
    \end{tabular}
    \label{tab:my_label}
\end{table}
the best linear unbiased estimate of $\alpha+\beta$ is equal to
\begin{multicols}{4}
\begin{enumerate}
    \item 1.5
    \item 1
    \item 1.8
    \item 2.1
\end{enumerate}
\end{multicols}

\item\label{1} Consider a discrete time Markov chain on the space \cbrak{1,2,3} with one-step transition  \begin{figure}[!ht] \centering
\resizebox{0.2\textwidth}{!}{%
\begin{circuitikz}
\tikzstyle{every node}=[font=\normalsize]
\node [font=\normalsize] at (1,10.25) {\textbf{1}};
\node [font=\normalsize] at (2,10.25) {\textbf{2}};
\node [font=\normalsize] at (3,10.25) {\textbf{3}};
\node [font=\normalsize] at (1,9.75) {\textbf{0.7}};
\node [font=\normalsize] at (2,9.75) {\textbf{0.3}};
\node [font=\normalsize] at (3,9.75) {\textbf{0}};
\node [font=\normalsize] at (1,9.25) {\textbf{0}};
\node [font=\normalsize] at (2,9.25) {\textbf{0.6}};
\node [font=\normalsize] at (3,9.25) {\textbf{0.4}};
\node [font=\normalsize] at (1,8.75) {\textbf{0}};
\node [font=\normalsize] at (2,8.75) {\textbf{0}};
\node [font=\normalsize] at (3,8.75) {\textbf{1}};
\node [font=\normalsize] at (0.25,9.75) {\textbf{1}};
\node [font=\normalsize] at (0.25,9.25) {\textbf{2}};
\node [font=\normalsize] at (0.25,8.75) {\textbf{3}};
\draw [ line width=0.9pt](3.25,10) to[short] (3.25,8.5);
\draw [ line width=0.9pt](0.75,10) to[short] (0.75,8.5);
\draw (0.75,8.5) to[short] (1,8.5);
\draw (0.75,8.5) to[short] (1,8.5);
\draw [ line width=0.5pt](0.75,8.5) to[short] (1.25,8.5);
\draw [ line width=0.5pt](0.75,10) to[short] (1.25,10);
\draw [ line width=0.5pt](2.75,10) to[short] (3.25,10);
\draw [ line width=0.5pt](2.75,8.5) to[short] (3.25,8.5);
\end{circuitikz}
}%
\label{fig{1}.1}
\caption{Probablity matrix}
\end{figure} probablity matrix. Which of the following statements is true ?
\begin{enumerate}
    \item States 1, 3 are recurrent and state 2 is transient.
    \item State 3 is recurrent and states 1, 2 are transient.
    \item States 1, 2, 3 are recurrent. 
    \item States 1, 2 are recurrent and state 3 is transient.\\
\end{enumerate}

\item The minimal polynomial of the matrix
$\myvec{1&1&2&0\\0&2&1&0\\0&0&1&0\\0&0&0&2\\}$ is
\begin{enumerate}
\begin{multicols}{4}
    \item $\brak{x-1}\brak{x-2}$
    \item $\brak{x-1}^2\brak{x-2}$
    \item $\brak{x-1}\brak{x-2}^2$
    \item $\brak{x-1}^2\brak{x-2}^2$
\end{multicols}
\end{enumerate}
\newpage
\item Let $\brak{X_1,X_2,X_3}$ be a trivariate normal random vector with mean vector $\brak{-3,1,4}$ and variance-covariance matrix $\myvec{4&0&0\\0&3&-3\\0&-3&4\\}$. Which of the following statements are true ?
\begin{enumerate}
\item $X_2$ and $X_3$ are independent.
\item $X_1+X_3$ and $X_2$ are independent.
\item $\brak{X_2,X_3}$ and $X_1$ are independent.
\item $\frac{1}{2}\brak{X_2+X_3}$ and $X_1$ are independent.
\end{enumerate}
\begin{enumerate}
\begin{multicols}{2}
     \item \brak{i} and \brak{iii}
     \item \brak{ii} and \brak{iii}
     \item \brak{i} and \brak{iv}
     \item \brak{iii} and \brak{iv}
 \end{multicols}   
\end{enumerate}
\item A $2^3$ factorial experiment with factors $A,B$ and $C$ is arranged in  two blocks of four plots each as follows: \brak{\text{Below,} \brak{1} \text{denotes the treatment in which $A,B$, and $C$ are at the lower level, ac denotes the treatment in which $A$ and $C$ are at the higher level and $B$ is at the lower level, and so on.}}
\begin{table}[h]
    \centering
    \begin{tabular}{|c|c|} 
        \hline
        \textbf{Column-I} & \textbf{Column-II} \\
        \hline
        P. Polyacrylonitrile & 1. Hard and brittle material \\
        \hline
        Q. Nylon-6,6 & 2. Very high temperature resistant polymer \\
        \hline
        R. Polytetrafluoroethylene (PTFE) & 3. H-bonding\\
        \hline
        S. Ebonite & 4. Acrylic fibre \\
        \hline
         & 5. Rubber \\
        \hline
          &6. Polyester fibre \\
        \hline
    \end{tabular}
    \label{tab:my_label}
\end{table}
The treatment contrast that is confounded with the blocks is
\begin{multicols}{4}
\begin{enumerate}
    \item $BC$
    \item $AC$
    \item $AB$
    \item $ABC$
\end{enumerate}
\end{multicols}

\item
Consider a fixed effects two-way analysis of variance model $Y_{ijk}=\mu+\alpha_i+\beta_j+\gamma_{ij}+\epsilon_{ijk}$, where $i=1, ... ,a;j=1, ... ,b;k=1, .. ,r$ and $\epsilon_{ijk}$
's are independent and identically distributed normal random variables with zero mean and constant variance. Then the degrees of freedom available to estimate the error variance is zero when
\begin{multicols}{2}
\begin{enumerate}
    \item $a=1$
    \item $b=1$
    \item $r=1$
    \item None of the above.
\end{enumerate}
\end{multicols}

\item For $k=1,2, ... ,10$, let the probability density function of the random variable $X_k$ be\\

\begin{center}
$f_{X_k}=$
$\begin{cases}
\frac{e^{-\frac{x}{k}}}{k}, & x>0\\ 
0, & \text{otherwise.}
\end{cases}$\\
\end{center}

Then $E\brak{\Sigma_{k=1}^{10} k X_k}$ is equal to ...
\end{enumerate}
\end{document}